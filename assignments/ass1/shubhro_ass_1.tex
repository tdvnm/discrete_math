\documentclass[11pt]{article}
\usepackage[utf8]{inputenc}
\usepackage[letterpaper,top=0cm, margin=0.85in]{geometry}

\usepackage{textcmds} %more symbols
\usepackage{fontspec} %more fonts

%for math
\usepackage{amsmath, amssymb, amsfonts} %standard
\usepackage{youngtab} % makes squares for math diagrams
\usepackage{microtype} %% <-- added
%-----------------------------------------------------------           

%\usepackage{sectsty}
%for lists and numbers
\usepackage{enumitem}
%-----------------------------------------------------------

% Doc setting
\usepackage[english]{babel} % Replace `english' with e.g. `spanish' to change the document language
\usepackage{setspace} %to set spacing bw words and lines
\usepackage{changepage}
% \setlength\parindent{0pt}

%footer
\usepackage{fancyhdr}
\usepackage{lastpage}

\fancyhf{}
\renewcommand{\footrulewidth}{0.2pt}
\renewcommand{\headrulewidth}{0pt} %remove headerline

% \fancyfoot[RE,RO]{\thepage}
\fancyfoot[L]{\textsc{comp201 - shubhro gupta}}
\fancyfoot[C]{\emph{Assignment 1}}
\fancyfoot[R]{\thepage}
\pagestyle{fancy}
%-----------------------------------------------------------

%for pictures and graphs
\usepackage{graphicx} %add image
\usepackage{adjustbox}

\usepackage{pgfplots} %for graphing plotting
\pgfplotsset{compat=1.18, width=10cm}
%-----------------------------------------------------------

%for code
\usepackage{verbatim}
\usepackage{listings}
\usepackage{fancyvrb} %for coding blocks
%\usepackage{algorithm}
%\usepackage{algpseudocode} %for pseudocode
%\usepackage{algorithm, algpseudocode}

%\usepackage{lstfiracode} %firacode
\usepackage[framemethod=tikz]{mdframed} %adding background to lstlisting
\usepackage[ruled,vlined,boxed]{algorithm2e} %for pseudocode lines



%for colors and links
\usepackage[colorlinks = true,
            linkcolor = blue,
            urlcolor  = blue,
            citecolor = blue,
            anchorcolor = blue]{hyperref}
\usepackage[many]{tcolorbox}  % for colored boxes
\usepackage{color} % to get colors
\usepackage{xcolor} %more colors options and flexibility
\usepackage{transparent}

\definecolor{background}{rgb}{0.16,0.16,0.21}
\definecolor{codegreen}{rgb}{0.24,0.68,0.65}
\definecolor{codepurple}{rgb}{0.51,0.31,0.87}
\definecolor{codered}{rgb}{0.81,0.13,0.18}
\definecolor{codebluegray}{rgb}{0.02,0.31,0.68}

\definecolor{comment}{rgb}{0.67,0.74,0.79}
\definecolor{textcolor}{rgb}{0.22,0.22,0.22}

%style for coding
\lstdefinestyle{python}{
    language=python,
    backgroundcolor=\color{white},   
    commentstyle={\color{comment}},
    keywordstyle={\color{codepurple}},
    stringstyle={\color{codegreen}},
    basicstyle={\ttfamily\color{textcolor}},
    keywordstyle = [2]{\color{codered}},
    keywordstyle = [3]{\color{codebluegray}},
    keywordstyle = [4]{\color{teal}},
    otherkeywords = {<, >, +, -, =, *, \[, \], &&, ||, format, 1, 2, 3, 4, 5, 6, 7, 8, 9, 0, ;},
    morekeywords = [4]{+, -, *, /, =, <, >, format},
    morekeywords = [3]{\[, \],  1, 2, 3, 4, 5, 6, 7, 8, 9, 0, ;},
    %
    breakatwhitespace=false, 
    % frame=shadowbox,
    rulecolor=\color{textcolor},
    breaklines=true,                 
    captionpos=b,                    
    keepspaces=true,                 
    % numbers=left,
    % numbersep=15pt, %distance between code and numbers
    % numberstyle=\scriptsize\ttfamily\color{comment},
    showspaces=false,                
    showstringspaces=false,
    showtabs=false,
    % xleftmargin=4.3em, %margin bw left page and frame
    % framexleftmargin=3.8em, %margin bw text and frame
    %xleftmargin=3.4em,
    % framexrightmargin=-0.5em,
    tabsize=2,
    % aboveskip=1.5em,
    % belowskip=0.5em,
    % framextopmargin=9pt,
    % framexbottommargin=9pt,
    %frames
    % frameshape={
}

\lstdefinestyle{c}{
    language=c,
    backgroundcolor=\color{white},   
    commentstyle={\color{comment}},
    keywordstyle={\color{codepurple}},
    stringstyle={\color{codegreen}},
    basicstyle={\ttfamily\color{textcolor}},
    keywordstyle = [2]{\color{codered}},
    keywordstyle = [3]{\color{codebluegray}},
    keywordstyle = [4]{\color{teal}},
    otherkeywords = {<, >, +, -, =, *, \[, \], &&, ||, stdio.h, stdlib.h, 1, 2, 3, 4, 5, 6,7, 8, 9, 0, ;},
    morekeywords = [4]{+, -, *, /, =, <, >, stdio.h, stdlib.h},
    morekeywords = [3]{\[, \],  1, 2, 3, 4, 5, 6, 7, 8, 9, 0, ;},
    morekeywords = [2]{&&, ||},
    %
    breakatwhitespace=false, 
    frame=shadowbox,
    rulecolor=\color{textcolor},
    breaklines=true,                 
    captionpos=b,                    
    keepspaces=true,                 
    numbers=left,                    
    numbersep=15pt, %distance between code and numbers
    numberstyle=\scriptsize\ttfamily\color{comment},
    showspaces=false,                
    showstringspaces=false,
    showtabs=false,
    xleftmargin=4.3em, %margin bw left page and frame
    framexleftmargin=3.8em, %margin bw text and frame
    %xleftmargin=3.4em,
    framexrightmargin=-0.5em,
    tabsize=2,
    aboveskip=1.5em,
    belowskip=0.5em,
    framextopmargin=9pt,
    framexbottommargin=9pt,
    frameshape={RYR}{Y}{Y}{RYR}
}

%-----------------------------------------------------------------------------
%custom commands

%code
\newcommand{\problem
}[2]{
\begin{mdframed}
    Exercise \textbf{#1} \hfill {1 \emph{Points.}}
\end{mdframed}
}
\newcommand{\codecap}[2]{{\vspace{4pt}{\emph{#1}}} \hfill \href{#2}{Link to the code\ }\vspace{25pt}}
\newcommand{\code}[1]{{\texttt{#1}}}

%math
\newcommand{\bigo}[1]{$O(#1)$ }
\newcommand{\thetan}[1]{$\theta(#1)$}
% \newcommand{\vector}[1]{$\overrightarrow{#1}$}

\newcommand{\vecset}[2]{\{ {#1}_1, {#1}_2, {#1}_3,  \dots,  {#1}_{#2}\}}

%display
\newcommand{\link}[3][blue]{\href{#2}{\color{#1}{#3}}}%
\newcommand{\inlink}[1]{\underline{\emph{\link[black]{#1}{#1}}}}


%header
\newcommand{\heading}[5]{
\begin{large}
\noindent\emph{#1}\smallskip ~\\
Professor #3 \hfill Week #2 \smallskip ~\\
\textbf{Shubhro Gupta} \hfill Due #4 ~\\
\end{large} \medskip ~\\
{\emph{Collaborators: #5}}~\\
\hrule
\vspace{50pt}
~\\
}

% \newcommand\dunderline[3][-1pt]{{%
%   \sbox0{#3}%
%   \ooalign{\copy0\cr\rule[\dimexpr#1-#2\relax]{\wd0}{#2}}}}

%new section
\newcommand{\asec}[1]{{\vspace{20pt}\large\dunderline[-3pt]{1pt}{\textbf{#1}}} ~\\}




%-----------------------------------------------------------------------------
%title
\usepackage{algpseudocode}
\begin{document}

\heading{Discrete Mathematics}{1}{T. V. H. Prathamesh}{6 September, 2024}{none}
\\
\section{question}
\emph{Given. } Murder occured involving father, mother, son and daughter. $M_1$ murdered $M_2$, $M_3$ was witness, $M_4$ was an accomplice.
\begin{itemize}
	\item $M_2$ and $M_4$ are opposite sex
	\item Oldest member and $M_4$ are opposite sex
	\item Youngest member and $M_3$ are opposite sex
	\item $M_4$ age > $M_2$ age
	\item Father was the oldest
	\item $M_1$ is not the youngest
\end{itemize}
\emph{To Find. } The murderer. \medskip \\
\emph{Solution. } \\













\section{question}
\emph{Given. } A pack of two-sided cards, claiming where if one side is vowel $\rightarrow$ other side is an even number. 4 cards are shown where we see \textbf{W, U, 4, 9}. \medskip \\
\emph{To Find. } The cards to turn to check if the conjecture is true or not. \medskip \\
\emph{Solution. } \\


\section{question}
\emph{To Negate. }
\begin{enumerate}
	\item $\forall n, \exists m \text{ s.t.  } n > m$ \\
	      $\exists n, \forall m \text{ s.t. } n \leq m$

	\item $\exists n, m \text{ s.t. } m \neq n$ \\
	      $\forall n,m \text{ st.  } n = m$

	\item $\forall m, m^2 > 0 \implies m > 0$ \\
	      $\exists m,m^2 \leq 0 \implies m > 0$

	\item $\forall a, b, c \in S, a\; R\; b \textrm{ and } b\; R\; c \implies a\; R\; c$

\end{enumerate}

\section{question}
\emph{Given. } We prove by induction that any $n$ things are the same. If $n = 0$ or $n = 1$ this is clear. or the induction step, assume we can say that any $k$ things are the same. Let a finite set of $k+1$ things, $x_1, x_2, x_3,\ldots x_k,x_{k+1}$ be given. By induction hypothesis, we get that if we take a subset of $k$ of these things, we get them to be the same. It follows that:
\begin{itemize}
	\item $x1 = x2 = ... = xk$
	\item $x2 = ... = x_k = x_{k+1}$
\end{itemize}
Thus it follows that \begin{align*}
	x_1 = x_2 = ... = x_k = x_{k+1}
\end{align*}
Thus for any $n$ things, they are the same.\medskip \\
\emph{To Find. } The error in the proof.

\section{question}
\emph{To Check and Prove. } Which of the following results hold.
\begin{enumerate}
	\item If $f$ is bijective and $g$ is bijective, then $g \circ f$ is bijective.
	\item If $f \circ f$ is injective, then $f$ is injective.
	\item if $f \circ f$ is surjective, then $f$ is surjective.
	\item If $g \circ f$ is surjective, then $f$ is surjective.
\end{enumerate}
\emph{Solution. } \\
\begin{enumerate}
	\item No. Consider $f(x) = x^2$ and $g(x) = x^3$. Then $f$ and $g$ are bijective but $g \circ f = x^6$ is not bijective.
	\item Yes. If $f \circ f$ is injective, then $f$ is injective. Let $f(x) = f(y)$. Then $f(f(x)) = f(f(y)) \implies x = y$.
	\item No. Consider $f(x) = x^2$. Then $f(f(x)) = x^4$ is not surjective.
	\item Yes. If $g \circ f$ is surjective, then $f$ is surjective. Let $y \in Y$. Then $\exists x \in X$ such that $g(f(x)) = y$. Since $g$ is surjective, $\exists z \in X$ such that $g(z) = y$. Thus $f(z) = x$.
\end{enumerate}








\section{question}
\emph{To Check. }If $f \circ g$ are bijective then are both $f$ and $g$ bijective? \medskip \\
\emph{Solution. } No. Consider $f(x) = x^2$ and $g(x) = x^3$. Then $f \circ g = x^6$ is bijective but $f$ and $g$ are not bijective.


\section{question}
\emph{To Show. } The set operation $A \textbackslash B$ is not associative. That is, $(A \textbackslash B) \textbackslash C \neq A \textbackslash (B \textbackslash C)$ \medskip \\
\emph{Solution. }


\vspace{50pt}
\noindent \emph{To Show. } Show that for every


\section{question}
\emph{To Prove. } $A \subset B$ and $B \subset C \implies A \subset C$. \medskip \\
\emph{Solution. }

\vspace{20pt}
\noindent \emph{To Prove. } $((P \longrightarrow R)\land(Q \longrightarrow R))\longrightarrow(P \lor Q)\longrightarrow R)$ \medskip \\
\emph{Solution. } \\
The truth table for the following is as follows \\
\begingroup
\centering
\begin{tabular}{|c|c|c|c|c|c|c|c|}
	\hline
	P & Q & R & $P \longrightarrow R$ & $Q \longrightarrow R$ & $P \lor Q$ & $(P \lor Q) \longrightarrow R$ & $((P \longrightarrow R)\land(Q \longrightarrow R))\longrightarrow(P \lor Q)\longrightarrow R)$ \\
	\hline
	T & T & T & T                     & T                     & T          & T                              & T                                                                                              \\
	T & T & F & F                     & F                     & T          & F                              & T                                                                                              \\
	T & F & T & T                     & T                     & T          & T                              & T                                                                                              \\
	T & F & F & F                     & T                     & T          & T                              & T                                                                                              \\
	F & T & T & T                     & T                     & T          & T                              & T                                                                                              \\
	F & T & F & T                     & F                     & T          & F                              & T                                                                                              \\
	F & F & T & T                     & T                     & F          & T                              & T                                                                                              \\
	F & F & F & T                     & T                     & F          & T                              & T                                                                                              \\
	\hline
\end{tabular}
\endgroup










\section{question}
\emph{Given. } $n \neq 0$ divides $k$ if there is a $q \in \mathbb{N}$ such that $k = n \times q$. \medskip \\
\emph{To Check. } Whether $divides$ as a relation on natural number is reflexive, symmetric, transitive or not. \medskip \\
\emph{Solution. } Let a relation $\sim$ defined on $\mathbb{N}$ as $\mathbb{N} \times \mathbb{N}$, such that $(k, n) \sim (n, k) \iff \exists q \in \mathbb{N}$ such that $k = n \times q$. \medskip \\



\section{question}
\emph{Given. } Panedmic affects 1 person on the first day. After that each person spreads it to $a$ people, where $a$  is average rate of transmission. Number of people infected on $n^{th}$ day is $(n-1)^{th} \times a$. Two programmers wrote the code below.
\medskip \\
\textbf{Naive Coder. }
\begin{lstlisting}[style=python]
fun pandemic(num-days :: Number, rate :: Number) -> Number:
    doc: "counting number of pandemic infected"
    if num-days == 0:
        1
    else:
        pandemic(num-days - 1, rate) * rate
    end
end
\end{lstlisting}

\noindent \textbf{Mathematics Coder. }
\begin{lstlisting}[style=python]
fun altpandemic(num-days :: Number, rate :: Number) -> Number:
    doc: "counting using power"
    num-expt(rate, num-days)
end
\end{lstlisting}
\vspace{10pt}
\emph{To Prove. } Given an input, both the functions give the same output. \medskip \\
\emph{Solution. }







\section{question}
\emph{To Write. } A bijective map from $\mathbb{N} \rightarrow \mathbb{Z}$. \medskip \\
\emph{Solution. } A bijection from $\mathbb{N} \rightarrow \mathbb{Z}$ can be given by the function $f(n) = (-1)^n \times \left\lceil \frac{n}{2} \right\rceil$.
\begin{align*}
	f(0) & = 0                                                  \\
	f(1) & = -1                                                 \\
	f(2) & = 1                                                  \\
	f(3) & = -2                                                 \\
	f(4) & = 2                                                  \\
	f(5) & = -3                                                 \\
	\vdots                                                      \\
	f(n) & = (-1)^n \times \left\lceil \frac{n}{2} \right\rceil
\end{align*}










\section{question}
\emph{Given. } A relation $\leq$ on a set $S$ is called a pre-order if it is reflexive and transitive. A relation $\equiv$ on $S$ such that $x \equiv y \iff x \leq y$ or $y \leq x$. \medskip \\
\emph{To Prove. } $\equiv$ is an equivalence relation. \medskip \\
\emph{Solution. } To prove that $\equiv$ is an equivalence relation, we need to show that it is reflexive, symmetric and transitive. \medskip \\








\section{question}
\emph{Given. } A pre-order is called a partial order if it is antisymmetric, i.e., $x \leq y \lor y \leq x \leftrightarrow x = y$ \medskip \\




\end{document}


