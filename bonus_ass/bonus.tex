\documentclass[11pt]{article}
\usepackage[utf8]{inputenc}
\usepackage[letterpaper,top=0cm, margin=0.85in]{geometry}

\usepackage{textcmds} %more symbols
\usepackage{fontspec} %more fonts

%for math
\usepackage{amsmath, amssymb, amsfonts} %standard
\usepackage{youngtab} % makes squares for math diagrams
\usepackage{microtype} %% <-- added
%-----------------------------------------------------------           

%\usepackage{sectsty}
%for lists and numbers
\usepackage{enumitem}
%-----------------------------------------------------------

% Doc setting
\usepackage[english]{babel} % Replace `english' with e.g. `spanish' to change the document language
\usepackage{setspace} %to set spacing bw words and lines
\usepackage{changepage}
% \setlength\parindent{0pt}

%footer
\usepackage{fancyhdr}
\usepackage{lastpage}

\fancyhf{}
\renewcommand{\footrulewidth}{0.2pt}
\renewcommand{\headrulewidth}{0pt} %remove headerline

% \fancyfoot[RE,RO]{\thepage}
\fancyfoot[L]{\textsc{comp201 - shubhro gupta}}
\fancyfoot[C]{\emph{Bonus Exam Questions 1}}
\fancyfoot[R]{\thepage}
\pagestyle{fancy}
%-----------------------------------------------------------

%for pictures and graphs
\usepackage{graphicx} %add image
\usepackage{adjustbox}

\usepackage{pgfplots} %for graphing plotting
\pgfplotsset{compat=1.18, width=10cm}
%-----------------------------------------------------------

%for code
\usepackage{verbatim}
\usepackage{listings}
\usepackage{fancyvrb} %for coding blocks
%\usepackage{algorithm}
%\usepackage{algpseudocode} %for pseudocode
%\usepackage{algorithm, algpseudocode}

%\usepackage{lstfiracode} %firacode
\usepackage[framemethod=tikz]{mdframed} %adding background to lstlisting
\usepackage[ruled,vlined,boxed]{algorithm2e} %for pseudocode lines



%for colors and links
\usepackage[colorlinks = true,
            linkcolor = blue,
            urlcolor  = blue,
            citecolor = blue,
            anchorcolor = blue]{hyperref}
\usepackage[many]{tcolorbox}  % for colored boxes
\usepackage{color} % to get colors
\usepackage{xcolor} %more colors options and flexibility
\usepackage{transparent}

\usepackage{enumitem}
\newlist{arrowlist}{itemize}{1}
\setlist[arrowlist]{label=$\hookrightarrow$}
%-----------------------------------------------------------------------------
%custom commands

%code
\newcommand{\problem
}[2]{
\begin{mdframed}
    Exercise \textbf{#1} \hfill \emph{Points }#2
\end{mdframed}
}
\newcommand{\codecap}[2]{{\vspace{4pt}{\emph{#1}}} \hfill \href{#2}{Link to the code\ }\vspace{25pt}}
\newcommand{\code}[1]{{\texttt{#1}}}

%math
\newcommand{\bigo}[1]{$O(#1)$ }
\newcommand{\thetan}[1]{$\theta(#1)$}
% \newcommand{\vector}[1]{$\overrightarrow{#1}$}

\newcommand{\vecset}[2]{\{ {#1}_1, {#1}_2, {#1}_3,  \dots,  {#1}_{#2}\}}

%display
\newcommand{\link}[3][blue]{\href{#2}{\color{#1}{#3}}}%
\newcommand{\inlink}[1]{\underline{\emph{\link[black]{#1}{#1}}}}


%header
\newcommand{\heading}[5]{
\begin{large}
\noindent\emph{#1}\smallskip ~\\
Professor #3 \hfill Bonus #2 \smallskip ~\\
\textbf{Shubhro Gupta} \hfill Due #4 ~\\
\end{large} \medskip ~\\
{\emph{Collaborators: #5}}~\\
\hrule
\vspace{50pt}
~\\
}

% \newcommand\dunderline[3][-1pt]{{%
%   \sbox0{#3}%
%   \ooalign{\copy0\cr\rule[\dimexpr#1-#2\relax]{\wd0}{#2}}}}

%new section
\newcommand{\asec}[1]{{\vspace{20pt}\large\dunderline[-3pt]{1pt}{\textbf{#1}}} ~\\}




%-----------------------------------------------------------------------------
%title
\usepackage{algpseudocode}
\begin{document}

\heading{Discrete Mathematics}{Assignment}{T. V. H. Prathamesh}{4 September, 2024}{none}
\\
\problem{1}{3}
\textbf{Given. } $f: \mathbb{N} \to \mathbb{N}$, such that, $\forall m, n \in \mathbb{N}, m < n \rightarrow f(m) < f(n)$.
$f(10) = 10$ \\
\noindent \textbf{To Prove. } $f(i) = i$ for $i\leq 10, n \in \mathbb{N}$.\\
\noindent \textbf{Proof by Induction. }\\
\emph{Base Case.}  $f(10) = 10$ (Given).\\
\emph{Inductive Hypothesis.}  Assume that $f(i) = i$ for $2 \leq i \leq 10$. We need to show that $f(i-1) = i-1$.\\
\emph{Inductive Step.}   Since $f$ is strictly increasing, $f(i-1) < f(i)$, and $f(i) = i$ (by inductive hypothesis). Therefore, $f(i-1) < i$. Since $f: \mathbb{N} \to \mathbb{N}$, $f(i-1)$ is a natural number. \medskip \\
The only natural number less than $i$ that $f(i-1)$ is equal, while still satisfying the condition $f(i-1) < i$, is $i-1$.
\begin{arrowlist}
	\item If $f(i-1) < i-1$, then it would not be a strictly increasing function. Since for some $m < i-1$, $f(m)$ has to be smaller than $f(i-1)$, which is not possible (there aren't enough distinct natural numbers to assign to $f(m)$ for $m<i−1$ while still maintaining the strictly increasing property).
	\item If $f(i-1) > i - 1$, then $f(i-1) \geq i$, which is a contradiction to the inductive hypothesis and the fact that $f(i-1) < i$.
\end{arrowlist}
Therefore, $f(i-1) = i-1$, and starting from $f(10) = 10$, we get $f(9) = 9$, $f(8) = 8$, and so on, till $f(1) = 1$. \hfill $\square$\\

\problem{2}{3}
\textbf{Given. } $f: \mathbb{N} \to \mathbb{N}$, such that, $\forall m, n \in \mathbb{N}, m > n \rightarrow f(m) < f(n)$. \\
\noindent \textbf{To Prove. } $\exists k \in \mathbb{N}, \forall n \in \mathbb{N}, n \geq k \rightarrow f(n) = 0$.\\
\noindent \textbf{Proof. }\\
The given condition states that $n$ increases $f(n)$ decreases. As $f(x): \mathbb{N} \rightarrow \mathbb{N}$, and is strictly decreasing, it cannot keep decreasing forever while remaining $\mathbb{N}$, i.e., $n < 0$, which is bounded below by 0, and the sequence must reach 0 at some point. \medskip \\
Let $k$ be the smallest natural number such that $f(k) = 0$.
Then, for any $n > k$, $f(n) < f(k) = 0$, and the only $\mathbb{N}$ less than 0 is 0. Therefore, $f(n) = 0$ for all $n \geq k$. \hfill $\square$\medskip \\
\emph{ It's not a strictly decreasing function and is a bit weird as the condition $m > n \rightarrow f(m) < f(n)$ can't really be fulfilled, since $f(n)=0$ always after some point. So we didn't even need to prove anything as the premise (if $f$ is strictly decreasing $\rightarrow \cdots)$ of the condition itself is not true.}\\



\problem{3}{3}
\textbf{Given. } $\exists e \in \mathbb{Z}, \forall x \in \mathbb{Z}, x + e = e$. $e_1$ and $e_2$ are two such integers such that $x + e_1 = x + e_2$ for all $x \in \mathbb{Z}$. \\
\noindent $x + e = e \rightarrow x = 0$, which implies that the only value that $x$ can take is 0. The statement seems incorrect because no integer $e$ can satisfy the condition $x+e=e$ for all integers $x$ unless the set of integers is restricted to a single value, which contradicts the definition of $\forall x \in \mathbb{Z}$.\\

\noindent \textbf{We need to redefine the problem statement.} That is $x+e = x$ for all $x, e \in \mathbb{Z}$, and $e_1$ and $e_2$ are two such integers such that $x + e_1 = x + e_2$ for all $x \in \mathbb{Z}$. \\

\noindent \textbf{To Prove. } $e$ is unique, $x + e_1 = x + e_2 \rightarrow e_1 = e_2$. $x=0$.\\
\textbf{Proof. }\\
\emph{Uniqueness.} Let $e_1$ and $e_2$ be two such integers such that $x + e_1 = x + e_2$ for all $x \in \mathbb{Z}$. Subtracting $x$ from both sides, we get $e_1 = e_2$. Therefore, $e$ is unique. \hfill $\square$\\
\emph{Existence.} Let $x = 0$, then $0 + e_1 = 0 + e_2 \rightarrow e_1 = e_2$. Therefore, $e = 0$. \hfill $\square$\\
\medskip \\
\emph{I don't understand this question at all, how is e unique? It's not unique, it's 0. Did we not need to prove this question as well? Since even here the premise itself is not true.}\\




% x$. $e_1$ and $e_2$ are two such integers such that $x + e_1 = x + e_2$ for all $x \in \mathbb{Z}$. \\
% \noindent \textbf{To Prove. } $e$ is unique, $x + e_1 = x + e_2 \rightarrow e_1 = e_2$. $e=0$.\\
% \noindent \textbf{Proof. }
% \begin{align*}
% 	x + e_1         & = x             & \text{(Given)}                \\
% 	(x + e_1) + e_2 & = x + e_2                                       \\
% 	x + (e_1 + e_2) & = x + e_2 \quad & \text{(Associative Property)} \\
% 	x + (e_1 + e_2) & = x             & \text{(Given)}                \\
% 	e_1 + e_2       & = 0             & \text{(Subtracting $x$)}      \\
% 	e_1             & = -e_2          & \text{(Subtracting $e_2$)}    \\
% 	x + e_1         & = x             & \text{(Given)}                \\
% 	x + (-e_2)      & = x             & \text{(Substituting $e_1$)}
% \end{align*}
% We can check this as $x + e_1 = x$ and $x + e_2 = x$ for all $x \in \mathbb{Z}$. Subtracting them we get $(x + e_1) - (x + e_2) = 0 \rightarrow e_1 - e_2 = 0 \rightarrow e_1 = e_2$. Therefore, $e_1 = e_2 = 0$, hence $e$ is unique. \hfill $\square$\\
%







\problem{4}{3}
\textbf{Given. } A relation $\sim$ defined on $\mathbb{Z}$ as $\mathbb{N} \times \mathbb{N}$, such that $(m_1, n_1) \sim (m_2, n_2) \iff m_1 + n_2 = m_2 + n_1$.\\
\textbf{To Prove. } $\sim$ is an equivalence relation.\\
\textbf{Proof }\\
\emph{Reflexive.} To show its reflexive, we need to show that $(m, n) \sim (m, n)$. $(m_1, n_1) \sim (m_1, n_1) \iff m_1 + n_1 = m_1 + n_1$. Since $m_1 + n_1 = m_1 + n_1$, $\sim$ is reflexive. \hfill $\square$
\medskip \\
\emph{Symmetric.} To show its symmetric, we need to show that $(m_1, n_1) \sim (m_2, n_2) \rightarrow (m_2, n_2) \sim (m_1, n_1)$. If $(m_1, n_1) \sim (m_2, n_2)$, then $m_1 + n_2 = m_2 + n_1$. Therefore, $m_2 + n_1 = m_1 + n_2$, which implies $(m_2, n_2) \sim (m_1, n_1)$. \hfill $\square$
\medskip \\
\emph{Tansitive.} To show its transitive, we need to show that $(m_1, n_1) \sim (m_2, n_2)$ and $(m_2, n_2) \sim (m_3, n_3) \rightarrow (m_1, n_1) \sim (m_3, n_3)$. If $(m_1, n_1) \sim (m_2, n_2)$, then $m_1 + n_2 = m_2 + n_1$. If $(m_2, n_2) \sim (m_3, n_3)$, then $m_2 + n_3 = m_3 + n_2$. Adding the two equations, we get $m_1 + n_2 + m_2 + n_3 = m_2 + n_1 + m_3 + n_2 \rightarrow m_1 + n_3 = m_3 + n_1$. Therefore, $(m_1, n_1) \sim (m_3, n_3)$. \hfill $\square$\\

\end{document}



















\end{document}


